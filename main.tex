%----------------------------------------------------------------------------------------
%	DOCUMENT DEFINITION
%----------------------------------------------------------------------------------------

% article class because we want to fully customize the page and not use a cv template
\documentclass[a4paper,12pt]{article}

%----------------------------------------------------------------------------------------
%	FONT
%----------------------------------------------------------------------------------------

% % fontspec allows you to use TTF/OTF fonts directly
% \usepackage{fontspec}
% \defaultfontfeatures{Ligatures=TeX}

% % modified for ShareLaTeX use
% \setmainfont[
% SmallCapsFont = Fontin-SmallCaps.otf,
% BoldFont = Fontin-Bold.otf,
% ItalicFont = Fontin-Italic.otf
% ]
% {Fontin.otf}

%----------------------------------------------------------------------------------------
%	PACKAGES
%----------------------------------------------------------------------------------------
\usepackage{url}
\usepackage{parskip} 	

%other packages for formatting
\RequirePackage{color}
\RequirePackage{graphicx}
\usepackage[usenames,dvipsnames]{xcolor}
\usepackage[scale=0.9]{geometry}

%tabularx environment
\usepackage{tabularx}

%for lists within experience section
\usepackage{enumitem}

% centered version of 'X' col. type
\newcolumntype{C}{>{\centering\arraybackslash}X} 

%to prevent spillover of tabular into next pages
\usepackage{supertabular}
\usepackage{tabularx}
\newlength{\fullcollw}
\setlength{\fullcollw}{0.47\textwidth}

%custom \section
\usepackage{titlesec}				
\usepackage{multicol}
\usepackage{multirow}

%CV Sections inspired by: 
%http://stefano.italians.nl/archives/26
\titleformat{\section}{\Large\scshape\raggedright}{}{0em}{}[\titlerule]
\titlespacing{\section}{0pt}{10pt}{10pt}

%for publications
\usepackage[style=iso-numeric]{biblatex}
%\usepackage[style=authoryear, sorting=ynt, maxbibnames=2]{biblatex}

%Setup hyperref package, and colours for links
\usepackage[unicode, draft=false]{hyperref}
\definecolor{linkcolour}{rgb}{0,0.2,0.6}
\hypersetup{colorlinks,breaklinks,urlcolor=linkcolour,linkcolor=linkcolour}
\addbibresource{citations.bib}
\setlength\bibitemsep{1em}

%for social icons
\usepackage{fontawesome5}

%debug page outer frames
%\usepackage{showframe}

%----------------------------------------------------------------------------------------
%	BEGIN DOCUMENT
%----------------------------------------------------------------------------------------
\begin{document}

% non-numbered pages
\pagestyle{empty} 

%----------------------------------------------------------------------------------------
%	TITLE
%----------------------------------------------------------------------------------------

% \begin{tabularx}{\linewidth}{ @{}X X@{} }
% \huge{Your Name}\vspace{2pt} & \hfill \emoji{incoming-envelope} email@email.com \\
% \raisebox{-0.05\height}\faGithub\ username \ | \
 %\raisebox{-0.00\height}\faLinkedin\ username \ | \ \raisebox{-0.05\height}\faGlobe \ mysite.com  & \hfill \emoji{calling} number
% \end{tabularx}

\begin{tabularx}{\linewidth}{@{} C @{}}
\Huge{Dana Suchomelová} \\[7.5pt]

\href{https://github.com/suchomelova-dana}{\raisebox{-0.05\height}\faGithub\ suchomelova-dana} \ $|$ \ 
\href{https://linkedin.com/in/dana-suchomelová-4a2014293}{\raisebox{-0.05\height}\faLinkedin\ Dana Suchomelová} \ $|$ \ 
%\href{https://linkedin.com/in/dana-suchomelová-4a2014293}{\raisebox{-0.05\height}\faGlobe\ mySite.cz} \ $|$ 
\href{https://dribbble.com/suchodan}{\raisebox{-0.05\height} Dribble} \ $|$ 


\href{mailto:danka.suchomelova@seznam.cz}{\raisebox{-0.05\height}\faEnvelope \ danka.suchomelova@seznam.cz} \ $|$ \ 
\href{tel:+420774878279}{\raisebox{-0.05\height}\faMobile \ +420 774 878 279} \\
\end{tabularx}

%----------------------------------------------------------------------------------------
% EXPERIENCE SECTIONS
%----------------------------------------------------------------------------------------

%Interests/ Keywords/ Summary
%\href{https://github.com/jitinnair1/autoCV}{click here}.
\section{O mně}
Jmenuji se Dana Suchomelová a zaměřuji se na vývoj frontendu a návrh uživatelského rozhraní. Vystudovala jsem Fakultu Informačních technologií Českého vysokého učení technického v Praze, obor Softwarové inženýrství. Studium mi dalo rozhled do mnoha sfér IT a položil všeobecné základy. Dále jsem se specializovala na návrh UI/UX, vývoj frontendu a související úkony. Preferuji minimalistický styl a jednoduchost, snažím se o uživatelskou přívětivost a konzistenci v návrzích.

%Projects
\section{Projekty}

Důležité odkazy:
 \href{https://dribbble.com/suchodan}{Dribble} - portfólio návrhů,
 \href{https://github.com/suchomelova-dana}{GitHub} - projekty


 \begin{tabularx}{\linewidth}{ @{}l r@{} }
\textbf{DBS portál} & \hfill \href{https://github.com/suchomelova-dana/Bachelor-Thesis}{GitHub} $|$  \href{https://www.figma.com/file/rjbyv2t9HdAaBuL6oOZsU1/Psan\%C3\%AD-test\%C5\%AF\?type=design\&node-id=278\%3A6835\&mode=design\&t=fDZXMS0PFdsedC3R-1}{Figma} \\[3.75pt]
\multicolumn{2}{@{}X@{}}{
DBS portál je webová aplikace Vysokého učení technického v Praze, která slouží pro výuku předmětu Databázové systémy. Na projektu jsem se začala podílet již ve 2. ročníku svého studia. Po dlouhodobé spolupráci jsem se poté rozhodla na projekt navázat v rámci své bakalářské práce. Práce zahrnovala sběr uživatelských požadavků, návrh,
implementaci a uživatelské testování.
Použité technologie: Figma, Vue.js 3 + Composition API, Typescript, Router, Pinia, ESLint, Quasar
}
\end{tabularx}




\begin{tabularx}{\linewidth}{ @{}l r@{} }
\textbf{Trenérský portál} & \hfill \href{https://github.com/suchomelova-dana/Trenersky-portal}{GitHub} $|$ \href{https://trenerskyportal.cz/}{trenerskyportal.cz} \\[3.75pt]
\multicolumn{2}{@{}X@{}}{
Trenérský portál je web vytvořený na zakázku. Jedná se o výukový portál pro trenérky moderní gymnastiky. Obsahuje edukativní videa. Portál byl vytvořen pomocí frameworku Vue.js a javascriptu.
}
\end{tabularx}




\begin{tabularx}{\linewidth}{ @{}l r@{} }
\textbf{Web SK TRASKO Vyškov} & \hfill \href{https://github.com/suchomelova-dana/SK-TRASKO-Vyskov}{GitHub} $|$ \href{https://www.figma.com/file/CQFsWhm8bDZLkdJ0MClBub/Sk-Trasko-Vy\%C5\%A1kov\?type=design\&node-id=0\%3A1\&mode=design\&t=fDZXMS0PFdsedC3R-1}{Návrhy Figma} \\[3.75pt]
\multicolumn{2}{@{}X@{}}{
Jedná se o statický web pro oddíl moderní gymnastiky SK TRASKO Vyškov. Návrhy byly realizovány ve Figmě a web byl implemntován pomocí Vue.js 3 a typescriptu. Web podporuje přihlášení členů oddílu pro soukromé informace. 
}
\end{tabularx}




\begin{tabularx}{\linewidth}{ @{}l r@{} }
\textbf{Youtube app} & \hfill \href{https://github.com/suchomelova-dana/Youtube-clone-App}{GitHub} \\[3.75pt]
\multicolumn{2}{@{}X@{}}{
Aplikace v Reactu inspirovaná aplikací Youtube. Uživatel v ní může sledovat videa, filtrovat a vyhledávat. Aplikace byla vytvořena v Reactu, a využívá online API. 
}
\end{tabularx}




\begin{tabularx}{\linewidth}{ @{}l r@{} }
\textbf{Psychologické cvičení - bludný kruh} & \hfill \href{https://www.figma.com/file/FFQuMzcxBgAm4RXJvZzPVw/Mindwell\?type=design\&node-id=0\%3A1\&mode=design\&t=fDZXMS0PFdsedC3R-1}{Figma} \\[3.75pt]
\multicolumn{2}{@{}X@{}}{
Jedná se o interaktivní návrh UI/UX ve Figmě pro mobilní i desktopové zařízení. Cvičení obsahuje bludný kruh (myšlenky, emoce, tělesné pocity a chování), který uživatel vyplňuje a má si tak uvědomit návaznost jednotlivých částí kruhu. 
}
\end{tabularx}


%----------------------------------------------------------------------------------------
%	EDUCATION
%----------------------------------------------------------------------------------------
\section{Education}
\begin{tabularx}{\linewidth}{@{}l X@{}}
2011 - 2019 & Gymnázium a Střední odborná škola zdravotnická a ekonomická Vyškov (víceleté gymnázium)
\end{tabularx}

\begin{tabularx}{\linewidth}{@{}l X@{}}	
2019 - 2023 & Bakalářské studium na FIT ČVUT | Softwarové inženýrství \\
\end{tabularx}

%----------------------------------------------------------------------------------------
%	SKILLS
%----------------------------------------------------------------------------------------

\section{Dovednosti}

\begin{tabularx}{\linewidth}{ @{}l X }
%--------------------------------------------------------
\multicolumn{2}{@{}X@{}}{\textbf{Programování}}\\

Javascript  \rule{0pt}{3ex} &
využíváno především s Vue.js pro vývoj frontendu webových aplikací, také pro statické weby\\

Typescript  \rule{0pt}{3ex} &
odbodně jako Javascript, Typescript jsem se naučila později, ale nyní ho preferuji\\

Vue.js 2, 3  \rule{0pt}{3ex} &
vyzkoušela jsem si obě verze frameworku, Vue 2 + Options API a Vue 3 + Composition API, přednost dávám Vue 3, zkušenost se souvisejícími knihovnami (Pinia, Quasar, Router, ESLint, I18n)\\

React \rule{0pt}{3ex} &
základní práce s frameworkem, využito pro osobní menší projekt\\

\texttt{C}, \texttt{C++} \rule{0pt}{3ex} &
základní znalost jazyků, použito pro školní projekty a semestrální práce\\

Scala  \rule{0pt}{3ex} &
základy programovacího jazyka, osvojení OOP patternů\\

Racket  \rule{0pt}{3ex} &
základy funkcionálního programování \\

Prolog  \rule{0pt}{3ex} &
základy logického programování\\

%--------------------------------------------------------
\\\multicolumn{2}{@{}X@{}}{\textbf{Návrh}}\\
Figma \rule{0pt}{3ex} &
hi-fi navrhování UI/UX pomocí Figmy, prototypování, vytváření komponent a jejich variant\\

Balsamiq Wireframes \rule{0pt}{3ex} &
lo-fi návrh rozhraní \\

Inkscape \rule{0pt}{3ex} &
využíváno pro vytváření loga a jiné vektorové grafiky \\

Gimp \rule{0pt}{3ex} &
úpravy fotek, retuš, koláže, bitmapová grafika \\

UML \rule{0pt}{3ex} &
class diagram, activity diagram, use case, sequence diagramy - pro modelování používán Enterprise Architect  \\

OntoUML \rule{0pt}{3ex} &
kategorie objektů, rigidita, vztahy celek-část  \\

Sběr požadavků \rule{0pt}{3ex} &
analýza a sběr uživatelských požadavků, kategorizace (FURPS, MoSCoW) \\





%--------------------------------------------------------
\\\multicolumn{2}{@{}X@{}}{\textbf{Ostatní}}\\

Git \rule{0pt}{3ex} &
vytváření a správa repozitářů, mergování, rebase...\\

Testování UI \rule{0pt}{3ex} &
uživatelské testování prostředí s testovacími subjekty, testovací případy\\

Vuepress \rule{0pt}{3ex} &
využito k vytvoření dokumentace frontendu webové aplikace\\

Databáze \rule{0pt}{3ex} &
konceptuální modelování, základní i pokročilejší dotazování SQL \\

Blender \rule{0pt}{3ex} &
základy 3D modelování \\

Jazyky \rule{0pt}{3ex} &
český jazyk (rodný), anglický jazyk (B1/B2)\\

\end{tabularx}



\vfill
\center{\footnotesize Poslední úprava: \today}

\end{document}